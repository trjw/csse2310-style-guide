\documentclass{article}
\usepackage[margin=2cm]{geometry}
\usepackage{syntax}
\usepackage{listings}
\usepackage{color}
\usepackage{tabularx}
\usepackage{wrapfig}
\usepackage{multicol}
\usepackage{fancyhdr}
\usepackage[right, displaymath, mathlines]{lineno}

%\usepackage{courier}
\setlength{\columnsep}{1cm}

% Create a nice looking tilde
\usepackage[hidelinks=true]{hyperref}
\newcommand{\propertilde}{\nolinkurl{~}}

% Setup style for programming excerpts
\definecolor{codegreen}{rgb}{0,0.6,0}
\definecolor{codegray}{rgb}{0.5,0.5,0.5}
\definecolor{codepurple}{rgb}{0.58,0,0.82}
\definecolor{backcolour}{rgb}{0.95,0.95,0.92}

\lstdefinestyle{mystyle}{
  basicstyle=\footnotesize\ttfamily,
  language=C,
  backgroundcolor=\color{backcolour},
  commentstyle=\color{codegreen},
  keywordstyle=\color{magenta},
  numberstyle=\tiny\color{codegray},
  stringstyle=\color{codepurple},
  basicstyle=\footnotesize\fontfamily{pcr}\selectfont,
  breaklines=true,
  showspaces=false,
  showstringspaces=false,
  stringstyle=\color{codepurple},
  tabsize=4,
  frame=single,
  captionpos=b,
  columns=fixed
}
\lstset{style=mystyle}
\renewcommand{\lstlistingname}{Example} % Listing -> Example

\title{CSSE2310/CSSE7231\\Documentation required for the use of AI tools\\Version 1.1}
\author{The University of Queensland\\School of Information Technology and Electrical Engineering}
\date{February 2024}
\fancypagestyle{plain}{
	\fancyhead{}
	\fancyfoot{}
	\fancyfoot[L]{\small{\copyright{~2024 The University of Queensland}}}
	\fancyfoot[R]{\thepage}
	\renewcommand{\headrulewidth}{0.5pt}
	\renewcommand{\footrulewidth}{0.5pt}
}
\fancypagestyle{fancy}{
	\fancyhead{}
	\fancyfoot{}
	\fancyhead[R]{CSSE2310/CSSE7231 -- Documentation required for the use of AI tools}
	\fancyfoot[L]{\small{\copyright{~2024 The University of Queensland}}}
	\fancyfoot[R]{\thepage}
	\renewcommand{\headrulewidth}{0.5pt}
	\renewcommand{\footrulewidth}{0.5pt}
}
\pagestyle{fancy}
\lfoot{\small{\copyright{~2024 The University of Queensland}}}


\begin{document}

\maketitle
\thispagestyle{plain}
\linenumbers

CSSE2310/7231 assignments \textbf{may} permit the inclusion of code generated by AI tools. If this is the case
then it will be clearly specified in the assignment specification. If such use is permitted and you do use such a tool, 
then you must reference the use in your code as described in the style guide. You must also fully document your use 
of the tool(s) in a text file that accompanies your submission.

\textbf{The use of such tools is NOT expected or required in this course and the teaching staff would prefer that you did
not use such tools as they may hinder your learning, may introduce bugs in your code, and it may actually take
you longer to complete an assignment than if you had not used the tool(s).}

In commercial environments, the use of such tools is being treated warily, if not disallowed altogether. The code that 
is generated may include snippets that are copied from other projects and are subject to particular licensing requirements, 
e.g. that the code can't be used in a commercial project, or that particular attribution is required, etc. It has also been found 
that the code generated by such tools can introduce subtle bugs and security issues into your work. 

If the use of such tools is permitted in a CSSE2310/CSSE7231 assignment and you decide to use one or more such tools, then it is your responsibility
to check the code that is generated actually does what you need it to AND to make sure you understand the code\footnote{You may be asked to explain and/or 
modify your code as part of the marking process. Inability to do so may result in failure of the assignment and/or misconduct 
allegations being made against you.} AND to appropriately reference/attribute the code in your source file(s) 
AND to document the use of the tool(s) fully.

Your documentation must be in a file named \texttt{toolHistory.txt} and committed to your repository (in the same directory as your 
source code for the assignment). The file must be committed to your repository with the first commit that contains any code generated
by such tools -- and must be kept up to date with all subsequent commits. You can not wait until your last commit before submission
to update the file -- you must keep it up to date as you work on the assignment.

It is entirely possible that the contents of this file will be much longer than the code itself. This process may also take you longer
than if you had just written the code yourself.

\section {General Guidelines}
Your \texttt{toolHistory.txt} file must be a plain text file that can be opened (and edited) with a text editor (e.g. Vim). You must include
timestamps to indicate when you used or interacted with the tool (or started interacting with the tool in a single session) and you must
specify which tool you were using. It expected that if you use a tool over multiple sessions that you just append details of later sessions
to those already in the file from earlier sessions.

You must capture all of your interactions with such tools that result in code or comments that you commit to your repository --
at the time you make those code commits. If you 
remove tool generated code from your work, then you may update your \texttt{toolHistory.txt} file to delete the details of the
tool interaction(s) that created that code and commit those changes at the same time (i.e. your \texttt{toolHistory.txt} file
always reflects the current commit). If you delete all such code you can remove the file from your repository also.
It is also acceptable if you maintain the full history of your tool interaction in the file, i.e. never delete anything, even if you 
remove tool generated code. The important requirement is that for any commit to your repository, the \texttt{toolHistory.txt} file
must contain at least the details of all tool interactions that have resulted in code or comments present in that commit.

It is expected that you might have to modify the code generated by such tool(s) to fix bugs or add features. 
You do not have to capture the details
of your own modifications to generated code here, but you should include a comment in your code reference (see the Style Guide)
that indicates that you modified the tool output. 

If you use such a tool to help you debug code that you wrote, and this results in you modifying your code, then 
you must capture the details in your \texttt{toolHistory.txt} file -- i.e. you have created/modified code because of an 
interaction with an AI tool, so the details must be captured.

If you have learned about a library function through interactions with such a tool then you must capture these details also.

There are no line length restrictions in the \texttt{toolHistory.txt} file,
but the file should be formatted in such a way that it is easy to read. 

\section{Tools That Capture Interactions}

For tools that capture your interactions with them (e.g. ChatGPT), then your \texttt{toolHistory.txt} file must include the full transcript(s) 
of your interaction(s) with the tool. There is no need to reformat this, and it is permissible for artefacts to appear if you copy and 
paste the conversation, e.g. the ``Copy code'' text  in the example below is from the ``Copy code'' button presented with code examples
in ChatGPT. You can add separators, blank lines etc. to make the file more readable if you wish.

For example:

\nolinenumbers
\lstset{language=,numbers=left,caption={ChatGPT code generation conversation example}}
\begin{lstlisting}
--------------------------------------
Wednesday 22 February 2023 2:48pm - interaction wtih ChatGPT
--------------------------------------
p.sutton@uq.edu.au
Write a C function that calculates the largest of the absolute values of two integer arguments.
Here's an example C function that takes two integer arguments and returns the largest of the absolute values of those integers:

c
Copy code
#include <stdlib.h>

int max_abs(int a, int b) {
    int abs_a = abs(a);
    int abs_b = abs(b);
    return abs_a > abs_b ? abs_a : abs_b;
}
This function first uses the abs() function from the standard library to calculate the absolute values of the input integers a and b. It then compares the absolute values using the ternary operator ? :, and returns the larger absolute value.

You can call this function with two integers as follows:

scss
Copy code
int result = max_abs(-10, 5);
This will set the result variable to the absolute value of -10, since it has a larger absolute value than 5.




p.sutton@uq.edu.au
Please rewrite the function without using the ternary operator.
Sure! Here's the same function as before, but without using the ternary operator:

c
Copy code
#include <stdlib.h>

int max_abs(int a, int b) {
    int abs_a = abs(a);
    int abs_b = abs(b);
    
    if (abs_a > abs_b) {
        return abs_a;
    } else {
        return abs_b;
    }
}
This function is identical to the previous one, except that it uses an if statement to check which absolute value is larger, and then returns the corresponding value.

You can call this function in the same way as before:

scss
Copy code
int result = max_abs(-10, 5);
This will set the result variable to the absolute value of -10, since it has a larger absolute value than 5.
\end{lstlisting}
\linenumbers

\section{Tools that Autocomplete}

For tools that provide auto-completion suggestions (e.g. Github Copilot), then you must capture all of:
\begin{itemize}
\item the context -- a brief description of the types and functions you have declared/defined so far (if this has not been
previously captured as the output from such a tool, e.g. if you wrote it yourself);
\item the prompt -- the text you have just entered at the moment just before you accept an autocomplete suggestion; and
\item the accepted suggestion.
\end{itemize}

\nolinenumbers
\lstset{language=,numbers=left,caption={GitHub Copilot code generation example}}
\begin{lstlisting}
--------------------------------------
Wednesday 22 February 2023 3:05pm - interaction with GitHub Copillot
--------------------------------------
Context
A file that has a single structure definition:
	struct nums {
     	   int a,b,c;
	};

Prompt
	// Function to return the largest member of struct nums
Accepted suggestion:
int largest(struct nums *n) {
        int max = n->a;
        if (n->b > max) max = n->b;
        if (n->c > max) max = n->c;
        return max;
}
\end{lstlisting}
\linenumbers

There is no need to show autocomplete suggestions that you do not accept. 

Capturing this information in appropriate detail is likely to be time consuming.
\end{document}